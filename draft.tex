\documentclass[letterpaper,twoside,12pt]{article}

  \usepackage{amssymb}

\usepackage{url}
\usepackage{hyperref}
\usepackage{graphicx}

% set margins to 1 inch
\usepackage[margin=1in, includehead, includefoot, headheight=15pt]{geometry}

% set indent to 0
\setlength{\parindent}{0pt}

% set paragraph spacing to 1em
\setlength{\parskip}{1em}

% set header and footer
\usepackage{fancyhdr}
\pagestyle{fancy}
\fancyhf{}
\fancyhead[LE]{\thepage \quad LaTeX Template for the Journal of Fair
Replication Studies}
\fancyhead[RO]{Name et al. \quad \thepage}
\renewcommand{\headrulewidth}{0pt}
\fancypagestyle{firstpagefooter}{
  \fancyhf{}
  \fancyfoot[C]{The Journal of Fair Replication Studies\\DOI: PENDING}
}
\fancypagestyle{plain}{
  \fancyhf{}
}

% define section format
\usepackage{titlesec}
\titleformat{\section}{\normalfont\Large\bfseries}{\thesection. }{0em}{}
\titleformat{\subsection}{\normalfont\large\bfseries}{\thesubsection. }{0em}{}
\titleformat{\subsubsection}{\normalfont\normalsize\bfseries}{\thesubsubsection. }{0em}{}

% Extras for Quarto

\usepackage{fancyvrb}
\usepackage{booktabs}
\usepackage{longtable}
\usepackage{multirow}
\usepackage{siunitx}

\usepackage{float}
\usepackage{xcolor}
\usepackage{framed}
\usepackage{calc}
\usepackage{longtable}
\usepackage{hyperref}

\providecommand{\tightlist}{%
  \setlength{\itemsep}{0pt}\setlength{\parskip}{0pt}}

\NewDocumentCommand\citeproctext{}{}
\NewDocumentCommand\citeproc{mm}{%
  \begingroup\def\citeproctext{#2}\cite{#1}\endgroup}
\makeatletter
 % allow citations to break across lines
 \let\@cite@ofmt\@firstofone
 % avoid brackets around text for \cite:
 \def\@biblabel#1{}
 \def\@cite#1#2{{#1\if@tempswa , #2\fi}}
\makeatother

\newlength{\cslhangindent}
\setlength{\cslhangindent}{1.5em}
\newlength{\csllabelwidth}
\setlength{\csllabelwidth}{3em}
\newenvironment{CSLReferences}[2] % #1 hanging-indent, #2 entry-spacing
 {\begin{list}{}{%
  \setlength{\itemindent}{0pt}
  \setlength{\leftmargin}{0pt}
  \setlength{\parsep}{0pt}
  % turn on hanging indent if param 1 is 1
  \ifodd #1
   \setlength{\leftmargin}{\cslhangindent}
   \setlength{\itemindent}{-1\cslhangindent}
  \fi
  % set entry spacing
  \setlength{\itemsep}{#2\baselineskip}}}
 {\end{list}}
\newcommand{\CSLBlock}[1]{\hfill\break#1\hfill\break}
\newcommand{\CSLLeftMargin}[1]{\parbox[t]{\csllabelwidth}{\strut#1\strut}}
\newcommand{\CSLRightInline}[1]{\parbox[t]{\linewidth - \csllabelwidth}{\strut#1\strut}}
\newcommand{\CSLIndent}[1]{\hspace{\cslhangindent}#1}

\newcommand{\VerbBar}{|}
\newcommand{\VERB}{\Verb[commandchars=\\\{\}]}
\DefineVerbatimEnvironment{Highlighting}{Verbatim}{commandchars=\\\{\}}
\definecolor{shadecolor}{RGB}{248,248,248}
\newenvironment{Shaded}{\begin{snugshade}}{\end{snugshade}}
\newcommand{\AlertTok}[1]{\textcolor[rgb]{0.94,0.16,0.16}{#1}}
\newcommand{\AnnotationTok}[1]{\textcolor[rgb]{0.56,0.35,0.01}{\textbf{\textit{#1}}}}
\newcommand{\AttributeTok}[1]{\textcolor[rgb]{0.77,0.63,0.00}{#1}}
\newcommand{\BaseNTok}[1]{\textcolor[rgb]{0.00,0.00,0.81}{#1}}
\newcommand{\BuiltInTok}[1]{#1}
\newcommand{\CharTok}[1]{\textcolor[rgb]{0.31,0.60,0.02}{#1}}
\newcommand{\CommentTok}[1]{\textcolor[rgb]{0.56,0.35,0.01}{\textit{#1}}}
\newcommand{\CommentVarTok}[1]{\textcolor[rgb]{0.56,0.35,0.01}{\textbf{\textit{#1}}}}
\newcommand{\ConstantTok}[1]{\textcolor[rgb]{0.00,0.00,0.00}{#1}}
\newcommand{\ControlFlowTok}[1]{\textcolor[rgb]{0.13,0.29,0.53}{\textbf{#1}}}
\newcommand{\DataTypeTok}[1]{\textcolor[rgb]{0.13,0.29,0.53}{#1}}
\newcommand{\DecValTok}[1]{\textcolor[rgb]{0.00,0.00,0.81}{#1}}
\newcommand{\DocumentationTok}[1]{\textcolor[rgb]{0.56,0.35,0.01}{\textbf{\textit{#1}}}}
\newcommand{\ErrorTok}[1]{\textcolor[rgb]{0.64,0.00,0.00}{\textbf{#1}}}
\newcommand{\ExtensionTok}[1]{#1}
\newcommand{\FloatTok}[1]{\textcolor[rgb]{0.00,0.00,0.81}{#1}}
\newcommand{\FunctionTok}[1]{\textcolor[rgb]{0.00,0.00,0.00}{#1}}
\newcommand{\ImportTok}[1]{#1}
\newcommand{\InformationTok}[1]{\textcolor[rgb]{0.56,0.35,0.01}{\textbf{\textit{#1}}}}
\newcommand{\KeywordTok}[1]{\textcolor[rgb]{0.13,0.29,0.53}{\textbf{#1}}}
\newcommand{\NormalTok}[1]{#1}
\newcommand{\OperatorTok}[1]{\textcolor[rgb]{0.81,0.36,0.00}{\textbf{#1}}}
\newcommand{\OtherTok}[1]{\textcolor[rgb]{0.56,0.35,0.01}{#1}}
\newcommand{\PreprocessorTok}[1]{\textcolor[rgb]{0.56,0.35,0.01}{\textit{#1}}}
\newcommand{\RegionMarkerTok}[1]{#1}
\newcommand{\SpecialCharTok}[1]{\textcolor[rgb]{0.00,0.00,0.00}{#1}}
\newcommand{\SpecialStringTok}[1]{\textcolor[rgb]{0.31,0.60,0.02}{#1}}
\newcommand{\StringTok}[1]{\textcolor[rgb]{0.31,0.60,0.02}{#1}}
\newcommand{\VariableTok}[1]{\textcolor[rgb]{0.00,0.00,0.00}{#1}}
\newcommand{\VerbatimStringTok}[1]{\textcolor[rgb]{0.31,0.60,0.02}{#1}}
\newcommand{\WarningTok}[1]{\textcolor[rgb]{0.56,0.35,0.01}{\textbf{\textit{#1}}}}

%%%%%%%%%%%%%%%%%%%

%%%%%%%%%%%%%%%%%%%%%%%%%%%
% Add additional packages

% \usepackage{pkgA}
% \usepackage{pkgB}
% etc

%%%%%%%%%%%%%%%%%%%%%%%%%%%

\begin{document}

\thispagestyle{firstpagefooter}

\begingroup  
\LARGE{\textbf{LaTeX Template for the Journal of Fair Replication
Studies}}\normalsize

\bigskip

  \textbf{You R. Name} \\
  \href{mailto:yrname@university.edu}{yrname@university.edu} \\
  \textit{University of Somewhere}\\
  \textbf{Jane A. Doe} \\
  \href{mailto:jadoe@university.edu}{jadoe@university.edu} \\
  \textit{University of Nowhere}\\
  \textbf{John B. Doe} \\
  \href{mailto:jbdoe@university.edu}{jbdoe@university.edu} \\
  \textit{University of Everywhere}\\

Received 15 October 2024; Revised PENDING; Accepted PENDING

\bigskip

\textbf{Abstract.} This is the abstract of the document.

Copyright \copyright 2024 Name, Doe and Doe. This is an open access
article distributed under the Creative Commons Attribution License,
which permits unrestricted use, distribution, and reproduction in any
medium, provided the original work is properly cited.
\endgroup

\section{OLS estimator}\label{ols-estimator}

The OLS estimator is given by \(\hat{\beta} = (X'X)^{-1}X'Y\), where
\(X \in \mathbb{R}^{n \times p}\) is the design matrix and
\(Y \in \mathbb{R}^n\) is the response vector (Hansen 2022). Therefore,
\(\hat{\beta} \in \mathbb{R}^p\).

A naive way to obtain the OLS estimator in R is:

\begin{Shaded}
\begin{Highlighting}[]
\NormalTok{b }\OtherTok{\textless{}{-}} \FunctionTok{solve}\NormalTok{(}\FunctionTok{t}\NormalTok{(X) }\SpecialCharTok{\%*\%}\NormalTok{ X) }\SpecialCharTok{\%*\%} \FunctionTok{t}\NormalTok{(X) }\SpecialCharTok{\%*\%}\NormalTok{ Y}
\end{Highlighting}
\end{Shaded}

\section{LM function}\label{lm-function}

A better way is to use the \texttt{lm} function:

\begin{Shaded}
\begin{Highlighting}[]
\NormalTok{n }\OtherTok{\textless{}{-}} \DecValTok{10}\NormalTok{L}

\FunctionTok{set.seed}\NormalTok{(}\DecValTok{123}\NormalTok{)}
\NormalTok{x1 }\OtherTok{\textless{}{-}} \FunctionTok{rnorm}\NormalTok{(n)}
\NormalTok{x2 }\OtherTok{\textless{}{-}} \FunctionTok{rbinom}\NormalTok{(n, }\DecValTok{1}\NormalTok{, }\FloatTok{0.5}\NormalTok{)}

\NormalTok{X }\OtherTok{\textless{}{-}} \FunctionTok{cbind}\NormalTok{(x1, x2)}

\NormalTok{y }\OtherTok{\textless{}{-}} \FunctionTok{rnorm}\NormalTok{(n)}

\FunctionTok{lm}\NormalTok{(y }\SpecialCharTok{\textasciitilde{}}\NormalTok{ X)}
\end{Highlighting}
\end{Shaded}

\begin{verbatim}

Call:
lm(formula = y ~ X)

Coefficients:
(Intercept)          Xx1          Xx2  
    -1.0249      -0.6145       0.9486  
\end{verbatim}

\section{LM function with rendered
table}\label{lm-function-with-rendered-table}

\begin{verbatim}
# A tibble: 3 x 5
  term        estimate std.error statistic p.value
  <chr>          <dbl>     <dbl>     <dbl>   <dbl>
1 (Intercept)   -1.02      0.747     -1.37   0.213
2 Xx1           -0.614     0.378     -1.62   0.148
3 Xx2            0.949     0.856      1.11   0.304
\end{verbatim}

\section{Plots}\label{plots}

\subsection{Ggplot2}\label{ggplot2}

\includegraphics{draft_files/figure-pdf/lm-plot-1.pdf}

\section{Acknowledgements}\label{acknowledgements}

The authors would like to thank the JRSSR team for providing this
template.

\section*{References}\label{references}
\addcontentsline{toc}{section}{References}

\phantomsection\label{refs}
\begin{CSLReferences}{1}{0}
\bibitem[\citeproctext]{ref-hansen2022econometrics}
Hansen, Bruce. 2022. \emph{Econometrics}. Princeton University Press.

\end{CSLReferences}

\end{document}
